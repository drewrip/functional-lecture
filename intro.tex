\documentclass{beamer}
\usepackage{listings}
\usepackage{graphicx}
\usepackage{amsmath}
\usetheme{Madrid}
\usecolortheme{seahorse}


\title{Intro to Functional Programming}
\author{Drew Ripberger}
\date{\today}

\begin{document}

\frame{\titlepage}

\section{introduction}

\begin{frame}
  \frametitle{Our Language of Choice}
  \begin{block}{Haskell}
    \textit{"it is a polymorphically statically typed, lazy, purely functional language, quite different from most other programming languages"}
    \\
    \alert{haskell.org}
  \end{block}
  \vspace{1cm}
  \includegraphics[width=.5\linewidth]{haskell-logo.pdf}
  \centering
\end{frame}


\begin{frame}
  \frametitle{Why learn functional programming?}
  \begin{itemize}
  \item Concise syntax
  \item Functional purity simplifies large codebases
  \item Recursion fits many problems extremely well
  \item Higher order functions are seen in numerous imperative languages
  \item More logically consistent with math
  \item Many functional concepts show up across languages
  \end{itemize}
\end{frame}

\begin{frame}
  \frametitle{The Changes with Functional}
  To state it simply, everything is a function.
  \begin{itemize}
  \item No variables
  \item No loops
  \end{itemize}
  Instead functions and recursion can be leaned on for all of these common tasks like iteration.
\end{frame}

\begin{frame}
  \frametitle{An Example}
  Counting down from 50
  \begin{block}{Haskell}
    \lstinputlisting[language=haskell]{examples/iter.hs}
  \end{block}
  \begin{block}{Python}
    \lstinputlisting[language=python]{examples/iter.py}
  \end{block}
\end{frame}

\begin{frame}
  \frametitle{Haskell Breakdown}
  \begin{block}{Haskell}
    \lstinputlisting[language=haskell]{examples/iter.hs}
  \end{block}

  Here \textit{mapM\textunderscore} is a function that maps the \textit{print} function to each of the elements of the list \textit{[50,49..1}

  This results in a countdown from 50 to 1 as such:
  \begin{texttt}\\
  50

  49

  48

  ...

  1
  \end{texttt}
\end{frame}

\end{document}