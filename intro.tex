\documentclass{beamer}
\usepackage{listings}
\usepackage{graphicx}
\usepackage{amsmath}
\usetheme{Madrid}
\usecolortheme{seahorse}


\title{Intro to Functional Programming}
\author{Drew Ripberger}
\date{\today}

\begin{document}

\frame{\titlepage}

\section{introduction}

\begin{frame}
  \frametitle{Our Language of Choice}
  \begin{block}{Haskell}
    \textit{"it is a polymorphically statically typed, lazy, purely functional language, quite different from most other programming languages"}
    \\
    \alert{haskell.org}
  \end{block}
  \vspace{1cm}
  \includegraphics[width=.5\linewidth]{haskell-logo.pdf}
  \centering
\end{frame}


\begin{frame}
  \frametitle{Why learn functional programming?}
  \begin{itemize}
  \item Concise syntax
  \item Functional purity simplifies large codebases
  \item Recursion fits many problems extremely well
  \item Higher order functions are seen in numerous imperative languages
  \item More logically consistent with math
  \item Many functional concepts show up across languages
  \end{itemize}
\end{frame}

\begin{frame}
  \frametitle{The Changes with Functional}
  To state it simply, everything is a function.
  \begin{itemize}
  \item No variables
  \item No loops
  \end{itemize}
  Instead functions and recursion can be leaned on for all of these common tasks like iteration.
\end{frame}

\begin{frame}
  \frametitle{An Example}
  Counting down from 50
  \begin{block}{Haskell}
    \lstinputlisting[language=haskell]{examples/iter.hs}
  \end{block}
  \begin{block}{Python}
    \lstinputlisting[language=python]{examples/iter.py}
  \end{block}
\end{frame}

\begin{frame}
  \frametitle{Haskell Breakdown}
  \begin{block}{Haskell}
    \lstinputlisting[language=haskell]{examples/iter.hs}
  \end{block}

  Here \textit{mapM\textunderscore} is a function that maps the \textit{print} function to each of the elements of the list \textit{[50,49..1]}

  This results in a countdown from 50 to 1 as such:
  \begin{texttt}\\
  50

  49

  48

  ...

  1
  \end{texttt}
\end{frame}

\begin{frame}
  \frametitle{Python Breakdown}
  \begin{block}{Python}
    \lstinputlisting[language=python]{examples/iter.py}
  \end{block}

  This Python code produces the same result, however in a vastly different way.

  This results in the same countdown from 50 to 1:
  \begin{texttt}\\
  50

  49

  48

  ...

  1
  \end{texttt}
\end{frame}

\begin{frame}
  \frametitle{Recursion in Functional}
  Determining if 22 shows up in a list of integers
  \begin{block}{Haskell}
    \lstinputlisting[language=haskell]{problems/has22.hs}
  \end{block}
\end{frame}

\begin{frame}
  \frametitle{The Same Problem Imperatively}
  \begin{block}{Python}
    \lstinputlisting[language=python]{problems/has22.py}
  \end{block}
\end{frame}

\begin{frame}
  \frametitle{Functions in Math and Haskell}
  \begin{block}{A Constant Function in Math}
    \[f(x)=5\]
  \end{block}
  \begin{block}{A Constant Function in Haskell}
    \[f=5\]
  \end{block}
\end{frame}

\begin{frame}
  \frametitle{Functions in Math and Haskell}
  \begin{block}{A Linear Function in Math}
    \[f(x)=2x\]
  \end{block}
  \begin{block}{A Linear Function in Haskell}
    \lstinputlisting[language=haskell]{examples/linear.hs}
  \end{block}
\end{frame}

\begin{frame}
  \frametitle{The Critical Observation}
  \begin{examples}
    \lstinputlisting[language=haskell]{examples/stringfunc.hs}
  \end{examples}
  \textbf{It is important to take notice of what this statement is saying.}

  \vspace{.5cm}
  
  In Python or any other imperative language, this would be saying that there exists a variable \textit{name}
  that has the string "Mel Hoffert" stored in it. However, in a functional setting, \textit{name} is a constant function.
\end{frame}

\begin{frame}
  \frametitle{In Summation}
  \begin{itemize}
  \item
  Functional programming has seemingly few \textit{technical} differences than any imperative language, 
  but, these few differences have mass implications. What you will notice is that while some of you Python experience will
  transfer to Haskell, having to program without side-effects, without loops, and without variables will completely change the way
  you have to approach a problem.
  \item
  Not knowing immediately how to solve what we would consider simple problems in a language will get frustrating, but just imagine you
  are back, a few years ago learning how to use an \textit{if} statement for the first time. Practice makes perfect.
  \end{itemize}
\end{frame}
\end{document}